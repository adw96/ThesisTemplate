\documentclass[phd,draft]{cornell}
%\documentclass[phd,tocprelim,draft]{cornell} 
% Options: phd/masters, draft/semifinal/final, tocprelim
% draft --> don't render figures
% tocprelim --> include preliminary sections in table of contents

\let\ifpdf\relax % not sure what this does...

%Some possible packages to include
\usepackage{graphicx}
%\usepackage[final]{graphicx} % add [final] to ensure that figures are printed even if \documentclass[draft] is on
%\usepackage[draft]{graphicx} %instead of setting entire \documentclass[draft], can just disable figure rendering like so:
%\includegraphics[draft=false]{image.pdf} %to render figure even when in draft mode, set it with this

\usepackage{palatino}
\usepackage{tabularx} 
\usepackage{threeparttable} %http://tex.stackexchange.com/questions/68605/table-with-foot-note since \tablenotemark{} is AGU specific!
\usepackage{longtable}
%\usepackage{amssymb}
%\usepackage{amsmath}
\usepackage{lscape}
\usepackage{rotating}
\usepackage{adjustbox} % for shrinking tables to fit on a page instead of rotating to landscape orientation

% Glossary 
% NOTE: if really ambitious you can generate a proper referenced index!
% see: http://texblog.org/2014/01/15/glossary-and-list-of-acronyms-with-latex/
%\usepackage{glossaries}
\usepackage[acronym,nomain,nonumberlist]{glossaries} 
%nomain allows definitions throughout text
%nonumberlist prevents page numbers from showing up in acronym list (index style), ok if only referencing acronym definition (first occurence with gls)
% Generate the glossary
\makeglossaries % comes before \begin{document}
% NEED TO RUN the following command line scripts to generate glossary/acronym index file so that it list shows up in final PDF:
% Simplified command (requires PERL): 
%makeglossaries Thesis
% Glossary
%makeindex -s Thesis.ist -t Thesis.glg -o Thesis.gls Thesis.glo
% List of acronyms:
%makeindex -s Thesis.ist -t Thesis.alg -o Thesis.acr Thesis.acn

% Turn labels and references into hyperlinks!!!!
%\usepackage{hyperref}
% AWESOME. But must compile with pdflatexmk

% Bibliography
\usepackage{natbib} 
\bibliographystyle{agu08}

%\usepackage{setspace} %singlespace by default
%\onehalfspacing
%\singlespacing
%\doublespacint

% Line numbering
%\usepackage{lineno}
%\linenumbers*[1]

%if you're having problems with overfull boxes, you may need to increase
%the tolerance to 9999
\tolerance=9999

% Set document spacing, margins, etc
\usepackage{caption} %required to make []{} short-long captions to work
%\usepackage{hangcaption} %non-standard default format for some reason in cornell.tex template
%\renewcommand{\caption}[1]{\singlespacing\hangcaption{#1}\normalspacing}
\renewcommand{\topfraction}{0.85}
\renewcommand{\textfraction}{0.1}
\renewcommand{\floatpagefraction}{0.75}

% ==================================================================================

% FRONT MATTER
% ----------------------------------
\title {My thesis title}
\author {My Name}
\conferraldate {May}{2017}
\degreefield {Ph.D.}
\copyrightholder{My name}
\copyrightyear{2017}

\begin{document}

\maketitle 
\makecopyright


% ABSTRACT
% ----------------------------------
\begin{abstract}
\centerline{\textbf{ABSTRACT}} %Labelling Abstract maybe not necessary...
What this thesis is about
\end{abstract}


% BIOGRAPHY
% ----------------------------------
\begin{biosketch}
How I came to write this thesis
\end{biosketch}


% ACKNOWLEDGEMENTS
% ----------------------------------
\begin{dedication}
To whom I dedicate this thesis
\end{dedication}

\begin{acknowledgements}
I acknowledge some people
\end{acknowledgements}


% TABLE OF CONTENTS
% ----------------------------------
\contentspage
\tablelistpage
\figurelistpage %NOTE: automatically includes full caption
%\listoffigures %is default graphicx command that is aware of short/long caption formatting
% see cornell.cls for other optional pages:
%\abbrlist %simple alternative to \glossaries
%\symlist %list of symbols

% More document spacing, margins, etc
\normalspacing \setcounter{page}{1} \pagenumbering{arabic}
\pagestyle{cornell} \addtolength{\parskip}{0.5\baselineskip}


% ACRONYM DEFINITIONS
% ----------------------------------
%\input{acronyms}
%Need to reference acronyms w/n sections, like {cvz}, {insar}, {snaphu}, for them to show up in glossary at the end.
% NOTE: could write these in separate files, and import them use \input{} and \include{} commands
% http://web.science.mq.edu.au/~rdale/resources/writingnotes/latexstruct.html
%\textbf{Appendices}\\
%Appendices include supplementary material related to each chapter including data tables and other auxiliary figures. 


% =======================================
% CH1 - INTRODUCTION
% =======================================
\chapter{My first chapter}
\label{Intro}

Some things you should know to start

\section{The first section}
\label{first}
To begin...
% NOTE: This section could be much better organized: WHAT make Uturuncu, Lazufre, and the CVZ unique?!
\subsection{The first subsection}
\label{utulaz}
Some more specific things to know

\subsection{A subsequent subsection}
How about this?

\section{The second section}
\label{GVD}
To continue...

% database updates log: http://www.volcano.si.edu/gvp_data_updates.cfm
\subsection{More subsections!}
A claim that...

\section{Narrative Outline}
Maybe a good idea to outline the chapters of this thesis

\textbf{Chapter \ref{Timeseries}}\\
In chapter 1 I do this

\textbf{Chapter \ref{Uturuncu}}\\
in chapter 2 I do this

\textbf{Chapter \ref{Lazufre}}\\
Chapter 3 is awesome!  I do this

\begin{table}
\begin{tabular}{cccccccc}
\hline
%1 & 2 & 3 & 4 & 5 & 6 & 7 & 8 \\
Satellite & Path\/Orbit & Geometry & \# Dates    & \# IFGs  & Date1      &  Date2     &  $\mathrm{B_{\perp}}$  \\
\hline
ALOS & 420  & Dsc.  & 4        & 3      & 2007/07/21 & 2008/01/21 & 690   \\
ALOS & 98  &  Asc.   & 17        & 49      & 2007/03/05 & 2011/03/16 & 2000   \\
ALOS & 99  & Asc.     & 14        & 42      & 2007/02/04 & 2011/02/15 & 2000   \\
TSX & 111\_009  & Dsc. & 6 & 8 	& 2012/06/10 & 2014/10/27 & 460  \\
\end{tabular}
\caption[A heading for this table]
{The caption for this table. These numbers mean nothing to me;  Doctor Scott did them for his thesis.}
\label{tab:newdata}
\end{table}



%\chapter{Glossary}
%NOTE: need to replace occurence of acronyms here in main text
%\printglossaries  %not sure if shows up in table of contents

%\chapter{More Acknowledgements}
% ---------------------------------------------------------
%In addition to the personal recognitions at the beginning of this document I include additional acknowledgments here:
% Moved it all to the beginning!



\end{document}
% =================================================
% =================================================